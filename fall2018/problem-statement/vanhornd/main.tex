\documentclass[draftclsnofoot,onecolumn,journal,letterpaper,10pt]{IEEEtran}

\usepackage{url}
\usepackage{geometry}
\usepackage{setspace}

\geometry{margin=0.75in}   
\singlespacing 




\begin{document}
\begin{titlepage}
    \title{CS 461 - Senior Capstone 
            \\Fall 2018
            \\Problem Statement
    }
    \author{Dan Van Horn \\ \small{vanhornd@oregonstate.edu}}
    \date{October 2018}
    \maketitle
    \begin{abstract}
Ninkasi Brewing Company currently relies on Excel spreadsheets and manual entry to track their brewing data and employees must manually enter  all of it. This process is not ideal for a company of this size and rate of growth as there has been data loss and it is time consuming to edit and manage. We are continuing development on an existing solution which includes a small set of applications designed to manage this data automatically and allow for manual user editing and visualization. This system will manage brewing data quickly, efficiently and correctly as it will reduce the possibility for human error. The applications will be hosted on a cloud provider and both desktop and mobile compatible.
    \end{abstract}
    \thispagestyle{empty}
\end{titlepage}


\section{Problem Statement}
Brewing data is recorded at Ninkasi by excel spreadsheets and other paper methods and is commonly shared through email. This process relies heavily on manual data entry which is time consuming and prone to error and can result in the loss of crucial data. The data stored in Excel is not structured in such a way to allow for any valuable brewing or business insight can be easily gained from it. The current growth rate of the company requires a systematic, automated approach to handling brewing data, and that it be comprehensive and accessible to more than a few people. In this way, as the company grows it can make informed decisions and streamline the brewing process. 

A working prototype exists from a previous capstone team but doesn�t fit the company�s needs yet. Some of the data history is not represented correctly, namely the history of each batch of beer brewed by type and the history of the fermenter itself and which batches have been brewed over its lifetime. The existing solution doesn�t keep track of what edits have been made to a batch of beer and no one other than administrators have this privilege. Non-administrators need the ability to do this and for that, each edit to a batch needs to be recorded to keep accountability. The data misrepresentation and lack of critical features translates into technical issues and bugs that will inevitably need to be fixed before we can move on to new feature implementation.



\section{Proposed Solution}

The fact that we are taking over an existing solution that was built by a previous capstone team puts us in a good place to meet the client�s needs. Less time will be spent on building the whole system so more time can be spent on refining it, which will result in a better product in the end. This solution is not currently used by Ninkasi and requires some fixes and new features. The first application is a website that is both mobile and desktop compatible, this reduces the overhead for maintaining both a mobile and desktop application. The site will display the status of the fermenters, automatically updating throughout the brewing process, as well as generating graphic visualizations of the data over time. The second application is an web server that interfaces with the site and securely provides it with brewing data if the user is properly authorized. It also communicates to a database which will contain all the brewing data and can efficiently create, read, update and delete entries. The site, server, and database will be their own separate nodes running in an environment hosted by a cloud provider that will be determined later.

There are problems with the design of the data that need to be addressed to more accurately represent the state of the brewing process as well as additional features that the company would like us to implement. We will improve the pathway by which data gets into the system as well as the usability of the website. We will implement a process by which a separate machine that Ninkasi uses for chemical analysis automatically feeds data to our system and expand on the set of graphic visualizations that can be generated for users.  The administrator controls for the site will be overhauled to make the system more flexible and non-administrator functionality will be created to increase the user base.

The requirements will be clarified and changed as the project progresses, but the overarching goal of the project is to get the system to a reliable state so that Ninkasi will decide to implement it. We will be implementing features on the technical side that make site maintenance minimal and adhering to modern software development practices to create a robust system that meets their needs. 



\section{Performance Metrics}
The intention with this project is to create a working prototype that can go into a beta stage of testing. The team from last year provided justification to Ninkasi that continued development is warranted and they have expressed the desire to continue and add more features that meet their business needs. 

The first milestone that will need to be completed will be to update the current system and fix all the bugs and issues that Ninkasi made us aware of. This stage will include creating development operations and testing procedures that will assist the team in adding future features. It will not directly impact the performance of the system but will reduce the time needed to implement the new features.

Next, we will be implementing the new features that were requested. At this point we expect to have the project specifications finely tuned and a thorough understanding of the complete system. We will remain flexible as requirements can, and most like will, change as time goes on. After the new feature implementation, we will be exploring the viable options for Ninkasi to do a beta release of the system and a cost benefit analysis on the different hosting options available.

The project will be complete when we have addressed Ninkasi�s concerns with the original system, the new features have been implemented, and the system has been deployed to a cloud provider. Depending on the completion of this stage, we may be able to add another phase to address feedback from Ninkasi again. It�s too early to speculate that far ahead, but we expect to have a lot of communication with Ninkasi throughout the year to make sure we are building the right system for them. 



\end{document}
