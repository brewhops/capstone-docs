\subsubsection{Fall}
\hfill\break
\noindent\medskip\textbf{Week 1}

Updated my resume and selected preferences for projects.

\noindent\medskip\textbf{Week 2}

We received the announcement that we were on the Ninkasi brewing operation team.
We sent out an email to Daniel to let him know we are on the team.
Coordinated to schedule a time for the team to travel to Eugene, visit the brewery and meet Daniel in person.

Created a slack channel for communicating amongst the team.

\noindent\medskip\textbf{Week 3}

Lily created the shared OneNote and GitHub for the team.
I joined the team GitHub and cloned the repository.
I wrote the first draft of the problem statement and uploaded my problem statement to the GitHub.
We visited Ninkasi Brewing and met with Daniel.
He gave us a tour of the brewery and looked at some of the systems they have in place for data tracking.
We also went though a typical workday for an employee to get more info on what kind of system would work well for them.
Gathered information on what kind of technological tools they have available.
Later in the week our team settled on a meeting time that the group and TA can agree on.

\noindent\medskip\textbf{Week 4}

We met with Andrew for the first time.
The meeting was short and to the point.
Andrew asked us what our project was, and we told him what we knew so far.
Andrew said the dangers for our project is having things to loosely defined.
The team had a meeting in the Library to discuss more details about the project and refine the problem statement as a group.
There was some writing on whiteboards, and the content was written down in the group OneNote document.
There is still issues getting everyone on the same tech.
We have the slack channel, our own personal OneNote's and the GitHub for us all, but I can't access the group OneNote and make any changes to it.

I wrote up most of the final draft of the problem statement.
It turns out that I took notes from the group one note under the general category, but Lily had written up most of the essay under the Fall tab and had not communicated to me that she had done that work.
We ended up getting two different versions of the final draft, which was then independently edited on several occasions.
This was a good lesson early on to be very talkative about what we were doing so that we didn't step on each-other or do work that had already been done.

\noindent\medskip\textbf{Week 5}

Began writing out a timeline for the project, which included a general breakdown with several stopping points for the project.
I reviewed the timeline with the group and transferred the timeline into a Gantt chart.

\noindent\medskip\textbf{Week 6}

Got feedback on the SRS, which was only an X on section 1.3.
Worked on filling out a bunch of sections for the new draft of the SRS. 25 commits in one day.
Changed the document so that every sentence is a new line which makes tracking on git way easier.
Created a rough draft of what a database design would look like with the information they have in Ninkasi's spreadsheet.
The version that was sent out to Daniel the first time was not the most up to date section and was missing all of section 3.6.
The first time he could look at it was Friday, so we needed to ask for an extension to make up both for the late review time from Daniel, and the problem where we sent the wrong draft in.

\noindent\medskip\textbf{Week 7}

Finished up the SRS and got it approved.
Had a phone call meeting with Daniel to iron out a few more details before we start on the Tech Review.
We organized who will be working on what section for the Tech Review.

\noindent\medskip\textbf{Week 8}

Wrote a rough draft version of the tech review, had it peer graded and then refined it through the week.
We got a clarification email that the tech review no longer became a group assignment.
We would start and finish our papers on our own.

\noindent\medskip\textbf{Week 9}

I spent several hours going in depth about the differences between React and Vue.
I'm sticking with Vue for now, but clarification from the client will help my decision.

\noindent\medskip\textbf{Week 10}

We had a meeting at Ninkasi.
The purpose was to talk with some employees over at deschutes brewery that began working on a project to bring some brewery management software to a Raspberry Pi framework.
The project seemed like a good idea, and we were worried that it would replace our product, but it turned out to be something targeted more to very small breweries that have just enough resources to begin tracking info about the brewing process.


\subsubsection{Winter}
\hfill\break
\textbf{Winter Break}

I did a large amount of work over winter break.
Before the fall term was over, I created prototypes of the interface and our group went down to Ninkasi to show them what we had made and got to iron out the details of the project one more time as a group before we began building.
In the first week of December, the team had a meeting discussing what we would cover before winter break started.
We wanted to get some more work done on the documents and make some revisions.
This was so we could send off everything to Daniel and give Ninkasi plenty of time over the break to work on revisions and feedback.
We set a plan to merge everything together in github and then make a revisions branch that we would work on.

Right before winter break started, I finished the UI designs and sent them off to the client so I could get some feedback.
I heard back on the 22nd of December with feedback both from the client and some of the brewers.
I implemented those changes and on the 27th, we had a video meeting for about an hour going over more improvements.
On the 30th, I got the changes back to the client.
We discussed meeting in person sometime early in the winter term to have one final round of updating and refining the UI.

\noindent\medskip\textbf{Week 1}

Though I had lots of time, I did not have a lot of work to do.
I had not yet gotten the chance to drive down to Ninkasi for the final feedback round, and I did not want to start building the UI without that feedback.

In the meantime, I focused on how our team would be working on the project, and learned how to implement Agile development on github.
The issues could be used as a task board, holding both tasks for current sprints and the backlog.
The milestones could be configured as sprints, though it would be up to the team to keep track of their sprints.
The Projects section would be where the tasks could be set in the to-do, in-progress, in-review and done sections.

\noindent\medskip\textbf{Week 2}

On January 15th, I went to Ninkasi on my own to get one final round of feedback on the UI.
I met a team of brewers and the client in the Ninkasi conference room and we took two hours to review the project design.
They made some additions that required a modification to both the UI and the database structure.
After the meeting, I stayed at the company until I finished implementing all the UI changes in the prototype that they had asked for.
I checked in with the client before leaving, and then set to work on building it.

Within the first week I had a solid layout for the website constructed out of just HTML and Stylus.
I put in about 10 hours of work between the 15th and the 20th, and had the app at an alpha level on a desktop environment.


\noindent\medskip\textbf{Week 3}

At this point, I worked on managing issues on GitHub, and helping Billy and Lily get their environments set up.
Lily’s computer had not been maintained well, so we eventually reinstalled the operating system to get her environment working smoothly.

\noindent\medskip\textbf{Week 4}

Billy needed some help getting rails to create the database from the templating code he had written.
I had been done with the UI for quite some time and Lily still had not started on the project, so I we agreed that I would take working on routing.
I needed routing implemented before I could appropriately build the UI for both desktop and mobile versions, so I took charge of routing so I could keep working on the project.

\noindent\medskip\textbf{Week 5}

This was another chunk of work for me.
I tried to get routing set up on the project and it just was not working.
It took about 6 hours for me to get unstuck on integrating routing into the UI I had already written.
We were using Gulp to automate the process of compiling the Stylus code, minifying all the files, and getting the site to run on localhost so all changes were immediately updated in the browser when a file is modified.
This was working fine for the UI I had written before, but with the integration of routing, I struggled with it long enough to figure out that its best that I switch the task runner from Gulp to Webpack.
This was not a decision I took lightly, and I did try to make gulp work for quite some time before I gave up.
Gulp is designed to be able to write some simple code to chain together a series of plugins.
These plugins were working well enough, but there were little things breaking pretty regularly and it seemed like I was building a dam out of oddly shaped puzzle pieces.

Switching to webpack involved tearing up the whole structure I had built and rebuilding it with a new template that was automatically generated by the Vue command line interface.
After 4 files modified, 21 files added, 31 files deleted and 3 renamed, I had mostly finished getting the new system working.

Once I got Webpack set up, I hooked up all the routing and got the site reacting appropriately with mobile devices connecting to pages specifically designed for small screens.


\noindent\medskip\textbf{Week 6}

I got the mobile and desktop pages to be much more reactive.
This means that the desktop version of the site is adaptable to any window size, and the site displays well on a mobile phone.
I built some special pages dedicated to, and only accessible on, mobile devices.

\noindent\medskip\textbf{Week 7}

This was another point where I would help Billy and Lily if they needed it, but I did not want to take their work.
So I did not put in any work this week.

\noindent\medskip\textbf{Week 8}

I met with Lily and we worked together on making some progress on the app.
One of the few problems the UI still had was that headers were available on the desktop version of the sites, but not on the mobile versions.
The mobile pages are simply constructed by taking components of the desktop page and breaking them into their own pages.
If the components had their own headers, there would be multiple headers on the desktop pages.
Lily and I worked together to fix this issue.

\noindent\medskip\textbf{Week 9}

I worked with Lily on making the username and password on the login require a match in the database before letting a user access the page.
I installed and implemented a cryptography JavaScript library to encrypt the users password through AES before sending it off to the database.

\noindent\medskip\textbf{Week 10}

Billy and Lily still have a lot of work to do, so I helped them when they needed it.
Other than that, I did all the paperwork required of the end of the term.

\subsubsection{Spring}
\hfill\break
\noindent\medskip\textbf{Week 1}

Implemented safety features ensuring accurate login and user authentication across the site, which involved setting a user as logged in with a cookie.
Added support for some minor additional features like being able to add variable amounts of information into the ingredients column on the recipes table in the database.

Had a meeting with Billy and Lily about clarifying some things and planning for what we need to get done this term.
We sent out an email to Daniel asking for a meeting time and some clarifications on data.
He got back to us and said that the 20th would work well for him and gave us some excel files that should have the data we are looking for.
Billy is going to make some more routes, Lily is going to be working on a lot of different things, and I’m going to get another burst of work done over the next week or two.


\noindent\medskip\textbf{Week 2}

Spent the day doing bug fixes and code cleanup.
Some of the notable activities were:

\begin{itemize}
  \item Removing placeholders
  \item Fixing indentation
  \item Adding number types to the HTML tags on the input fields
  \item Setting the password fields to hide user input
  \item Cleaning up and simplified the data submission process to the routes
  \item Adding the ability to create a new user working
\end{itemize}

\noindent\medskip\textbf{Week 3}

Converted the charts on the tank display page into a Vue component to make it easier to generate the charts for similar data.
Did some code cleanup.
The coolest thing was finally getting the option to switch between production and development environments.
This means that each developer could have their own version of the database for testing.

The team drove down to Ninkasi to do a presentation of the final product.
We met with Daniel and two of the brewers and presented both the desktop and mobile version of the app.
They said they loved the app and had a few additional requests.
The biggest one was to have a page where they could get information on a batch that had been closed.
We discussed the end of the project and what our team and our client needed to do to finish out the term.

\noindent\medskip\textbf{Week 4}

Created a simple batch history page that allows the user to download the entire history of a batch.
This is downloaded as a CSV file, with lots of customization regarding the organization of the information in the file and the custom name to make tracking multiple downloaded files easier.

\noindent\medskip\textbf{Week 5}

Finished the poster design, submitted for review and submitted the poster for printing

\noindent\medskip\textbf{Week 6}

I sent an email to Daniel letting him know that the app was ready for him and his team to work on.
This was our beta release of the project, and no more work on the software will be released after this point.

\noindent\medskip\textbf{Week 7}

Made a few little changes to the user interface to keep any weird bugs from showing up during expo.
Those changes were inconsequential to the app and kept the UI from doing weird things when people enter in really big numbers.
Senior Capstone expo.
We talked to a few people from some industry like eBay, Nvidia and HP and they were pretty excited about either the project our execution.

\noindent\medskip\textbf{Week 8}

No work done this week.

\noindent\medskip\textbf{Week 9}

Submitted the Final Progress Report Video

\noindent\medskip\textbf{Week 10}

Created a template for the final report, divided up the tasks amongst the group and worked on completing my tasks.
