\subsubsection{Fall}
\hfill\break
\noindent\medskip\textbf{Week 1}
We waited on official project confirmation from Kevin.
We also talked with Daniel Sharp about submitting a project for the capstone class.

\noindent\medskip\textbf{Week 2}
We planned a meeting at Ninkasi with Daniel Sharp to tour the facilities and talk about specific requirements.
We created communication channels (slack, email threads).

\noindent\medskip\textbf{Week 3}

We started and finished the Problem Statement first draft.
We also created our Github repository and uploaded our Problem Statement.
Moving forward, all of our version control took place through Github.
On Wednesday, we met with Daniel Sharp at Ninkasi, got a tour, saw the current technologies in place, and saw Ninkasi's current "database" system.
We discussed how we could replace the excel spreadsheet with a standalone database and the type of information to store.

\noindent\medskip\textbf{Week 4}

We had a communication issue on final draft of the problem statement, which was resolved.
We learned to have better communication and be clear about where draft development should take place.
The problem statement was finished this week and sent off to Daniel Sharp for approval.
We asked Daniel for a copy of the spreadsheet this week so we could begin designing our database.
Our first meeting with Andrew took place.
There has not been a lot to discuss during these meetings, so they have been similar to the first meeting.

\noindent\medskip\textbf{Week 5}
We finished the Specific Requirements Document first draft.
It was incomplete, but we had many questions for Daniel Sharp that needed to be addressed before we could determine the project requirements.
Daniel sent us confirmation of the Problem Statement document final draft.
This week in our TA meeting, Andrew told us about meeting attendance policies and proper folder format within the Github repo.


\noindent\medskip\textbf{Week 6}

We had a significant issue when we sent Daniel Sharp an incomplete version of the Specific Requirements Document, which Daniel reviewed Friday.
We realized our mistake and asked for extension and got approval to turn our document in Monday.
We waited for Daniel’s approval of actual complete final draft to turn in Monday of Week 7.
Earlier that week on Wednesday, we called Daniel to set up a meeting time with another brewery trying to implement a similar system.
We also discussed questions we had about Daniel’s take on the Specific Requirements Document.

\noindent\medskip\textbf{Week 7}
This week we received official confirmation on the Specific Requirements Document final draft from Daniel Sharp.
We divided up the Technology Review in components that each teammate could complete.
Connor took on front end technologies and data visualization software, Billy took on the database format and whether our project would be a native app or web app, Lily took on back end technologies and whether we would use a physical server or cloud hosting.

\noindent\medskip\textbf{Week 8}

This week we got our first drafts of the Technology Reviews and got them reviewed in class by our peers.

\noindent\medskip\textbf{Week 9}

We completed our Technology Review and determined our technology stack.
We had a problem when our chosen technologies conflicted with other components (see problem 3 in above Problems section).
Finally, we decided on cloud hosting of our PostgreSQL database, Ruby on Rails as our backend technology, and Apache as our web server.
On the front end we decided on HTML, SCSS, and Vue.js for the mobile-web user interface and C3.js for data visualization.

\noindent\medskip\textbf{Week 10}
We finished our Design Document and started our Progress Report.
On Friday, we met with Daniel Sharp at Ninkasi and had a Skype call with Deschutes Brewery who is trying to implement a similar system.
We learned about their database and website (more customizable than ours, but less user friendly) and that there is a great need for this kind of software in the craft brewing industry.
Daniel, some brewing staff, and the Business Intelligence staff were all thrilled with our progress and plans.
With their input and help, we clarified some details on how they would like to interact with the interfaces and database.
We created the Progress Report document and presentation at the end of this week.

\subsubsection{Winter}
\hfill\break
\noindent\medskip\textbf{Week 1}
This week we started off slow for me.
Connor started working on UI immediately.
Our plan is to have the UI done (Connor is close already) before we dive in to middle stack work.
That way there isn’t conflicting code for the website skeleton and Connor’s outline of the website and css are streamlined and clean.

\noindent\medskip\textbf{Week 2}
I’m learning Vue and I’m having lots of syntax errors.
I understand the general format of javascript because it’s similar to a lot of imperative languages I’ve used before, but the tiny syntax details of Vue has me confused.
I am working on creating my own tiny websites to understand how the general Vue object works and how to change data from the page.

\noindent\medskip\textbf{Week 3}

Still learning Vue, but now focusing on the bindings, v-ifs, and v-fors.
The templated code that is then implemented as tags is very handy but I can’t quite see how it’s going to be utilized for larger elements.
For example, we want the form and the tank monitoring to be separate components that we can either display on the same screen on bigger devices, or display one at a time on tablets or mobile phones.
I started error handling based off some code Connor created last term for a personal project.
Seeing implementations of Vue error handling really helped me understand what I needed to do.
\noindent\medskip\textbf{Week 4}
I had to factory reboot my computer this week as it was incredibly slow and some of the technology needed for locally building the website wasn’t able to download on my computer.
This set me behind a bit as I had to spend a lot of time setting up everything on my “new” computer (it’s amazing how many passwords I don’t have memorized).
I continued familiarizing myself with form validation and practicing on the login form.

\noindent\medskip\textbf{Week 5}

Connor ripped up the site this week to reformat it and it’s a lot better for routing now, but I need to relearn how to build the site and move the work I did in form validation to the new site version.
I finished lots of form validation and am getting better at programming in Vue.
I encounter less syntax errors, can code faster, and have to look up things less often (still very often though).
\noindent\medskip\textbf{Week 6}

Crazy busy week, therefore not as much done on the Vue side as I’d like.
I finished error checking for the login but I’m having trouble actually rerouting the page.
I probably will need to use another method because I think there is an issue with binding a variable before I try to edit it, therefore I can’t prevent the login submit from redirecting the page.

\noindent\medskip\textbf{Week 7}
This week I finished the login by force routing to another page, rather than setting the routing path and trying to prevent it if the login information was incorrect.
This worked instantly and I was able to move on to pulling tank information from the REST API.
Since there was no dummy data, I had to install and learn Insomnia to fill in values.
I was able to pull the data at the end of my work session and display the database info.
I can parse the information and compare to user input which allowed me to finish the login capabilities.

\noindent\medskip\textbf{Week 8}
This week was a slow work week. I had a 2-hour research presentation at the end of the week and a bunch of immediate deadlines in my classes, so the stress and time-constraints weren’t great facilitators of capstone productivity.
I was able to work for a bit on Saturday and Sunday with Connor to start entering new login/employee information, figuring out what information we needed from Billy to start pulling information from tanks, and general error handling.

\noindent\medskip\textbf{Week 9}
A lot of work was done this week, partially because now we had more time to dedicate to capstone, and also because at this point we are familiar enough with our technologies that using and updating our code is easy and quick.
You can now enter a new employee (there's an issue with password validation when you are creating it for the first time), a new recipe (the recipe is not segmented into ingredients as that requires another ingredient table which is yet to be implemented).

\noindent\medskip\textbf{Week 10}
On this last week of winter term, we finished up a few details so that the code and visual displays are cleaned up for the video demo.
We met on Thursday to discuss what needed to happen before our video demo and what needs to happen over break.
On Saturday, I finished the display-tank functionality which, out of the work I was in charge of, made the largest visual impact on our site.
We demoed the code and created our video on Saturday.

\subsubsection{Spring}
\hfill\break
\noindent\medskip\textbf{Week 1}
This week I hit the ground running.
I realized in the last week of the term I wasn't as close to beta as my team members and really needed to get my butt in gear.
I am a lot more competent with Vue now so work goes faster.
I finished creating a new tank, adding a new recipe, login form validation as well as changing input fields.
Now there are dropdowns for tanks as well as batches.


\noindent\medskip\textbf{Week 2}
This week was also really busy in terms of cranking out code.
Now the placeholders for the input fields are dynamic and change when you select a tank from the dropdown.
The input placeholders are auto-filled with the most recent tank data.
I also fixed the problem we were having where after inputting information about a tank it would switch back to the first inputted tank.
I also have the functionality to display a single tank and its information.

\noindent\medskip\textbf{Week 3}
This week we finished up so we could meet with Daniel on Friday.
I made a create action form, the ability to display most recent info from batches (which was buggy before), and the ability to submit tankId made by the user.
I also fixed the display so now the chosen names for tanks and batches are shown rather than the database generated ids.
We visited Daniel on Friday and had great success showing our product.
They will test our beta version and we will fix any bugs they point out.
Depending on funding, Connor and Billy will continue work over the summer.


Visit with Daniel on friday
\noindent\medskip\textbf{Week 4}
We fixed the few issues Daniel wanted changed: setting admin upon user creation and displaying actions with the correct colors.
We submitted the wired article this week. I interviewed Robert about his project, the Smart Thunder Lamp.


\noindent\medskip\textbf{Week 5}
This week we were pretty slow because of the rush from the previous three weeks.
We submitted an edited poster to be printed for expo, and we created a midterm report and presentation.

\noindent\medskip\textbf{Week 6}
Not a lot of work done this week other than prep our site for Expo.
Connor made a few fixes and it is looking really snazzy.

\noindent\medskip\textbf{Week 7}
Expo happened this week and it was exciting as well as overwhelming.
Our booth was immensely popular and we got feedback from several people that we gave great presentations to the visitors.
Our site is being tested at Ninkasi and we will soon hear back from them about any bug fixes.
They mentioned wanting to continue the project with another capstone team next year so I talked to Henry about his group's interest in the project.

\noindent\medskip\textbf{Week 8}
I leave to Carnegie Mellon at the end of this week so I finished up my parts of the project and let everyone know I was leaving.
I will have to do some documentation from Pittsburgh but it will be easy to handle as almost all of my work is online or has been delivered to my team.

\noindent\medskip\textbf{Week 9}
NA

\noindent\medskip\textbf{Week 10}
I am finishing compiling the documents so we can create a final report of our work in Capstone.
Connor submitted the final video this weekend, so we are all working together to finish this last document tonight.
